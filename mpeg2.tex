\documentclass[journal]{IEEEtran}
% *** GRAPHICS RELATED PACKAGES ***
\ifCLASSINFOpdf
  \usepackage[pdftex]{graphicx}
\else
  \usepackage[dvips]{graphicx}
\fi
\usepackage{float}
% *** MATH PACKAGES ***
\usepackage[cmex10]{amsmath}

\begin{document}
%
% paper title
\title{Benchmark Decodificaci\'{o}n MPEG2}
\author{Alberto~Suarez,~\IEEEmembership{Ingeniero,~Hewlett~Packard,}
        Yu~Shan~Hsieh,~\IEEEmembership{Ingeniero,~Hewlett~Packard}% <-this % stops a space
\thanks{}% <-this % stops a space
}
% make the title area
\maketitle

% in the abstract or keywords.
\begin{abstract}
El tama\~no f\'isico de las RUUs tienen mucho impacto sobre el desempe\~no. Si se utiliza un RUU relativamente peque\~no el CPI puede verse afectado negativamente produciendo un pobre resultado de funci\'on costo, perdiendo as\'i, la leve
mejora en consumo de potencia que pudo haber obtenido.
\end{abstract}

% Note that keywords are not normally used for peerreview papers.
\begin{IEEEkeywords}
Decodificaci\'{o}n, mpeg2, Benchmark
\end{IEEEkeywords}

\section{Introducci\'{o}n}
\IEEEPARstart{E}{l} prop\'{o}sito de este paper es benchmark sobre un programa de decodificaci\'{o}n \
de mpeg2, para este prop\'osito se va a analizar el mismo para crear una configuraci\'on inicial de \
Hardware para ser utilizada como referencia a la hora de correr el benchmark. Las posibles variables \
o parametros a variar en el benchmark son consumo de potencia, desempe\~{n}o de la CPU, tiempo de ejecuci\'{o}n, etc. 
Se realizar\'a una corrida premilimar del codigo para determinar el tiempo requerido para que se \
ejecute una iteraci\'on del c\'odigo, a partir del cua\'al se calcular\'a la cantidad de iteraciones \
que se pueden correr en un lapso sugerido de 45 minutos. \newline  Nos vamos a basar en la cantidad \
calculada de iteraciones para determinar la cantidad de par\'ametros a controlar a la hora de correr \
el testbench. Posteriormente de detallar\'{a} las pruebas que se van a realizar, se proceder\'a a \
correr el benchmark, para lo cual se va a utilizar un script que lo ejecute con los par\'ametros \
seleccionados y recolecte los datos pertinentes para su posterior an\'alisis. Luego se har\'{a} el \
an\'{a}lisis de resultados y finalmente las conclusiones que se pueden derivar de estos experimentos.


\hfill Setiembre 27, 2013

\section{Metodolog\'{i}a de optimizaci\'{o}n}

\subsection{Descripci\'{o}n del benchmark}
El decodificador de mpeg2 viene como parte de un set de benchmarks en el archivo comprimido \textit{mediabench.tar.gz}. Al descomprimir el archivo, se va a encontrar el directorio \textit{mpeg2}. Este es un benchmark que trabaja tanto la parte de codificaci\'{o}n (encode) como la de decodificaci\'{o}n. En nuestro caso, solo vamos a trabajar la parte de decode port lo que podemos ignorar todo lo relacionado con encode.

Dentro del directorio principal podemos encontrar los siguientes directorios que valen la pena mencionar \newline

\begin{itemize}
\item \textbf{src/mpeg2dec}. Contiene el codigo en C del decodificador y el Makefile.
\item \textbf{bin/}. Contiene los binarios creados por el Makefile.
\item \textbf{exec/}. Contiene scripts con demostraciones sencillas. Utiliza los binarios localizados en \textit{bin/}.
\item \textbf{data/}. Contiene los archivos temporales creados y utilizados por los scripts de prueba.
\end{itemize}

Dentro del directorio de src/mpeg2dec se encuentra el codigo en C del decoder junto con el Makefile. El archivo principal es mpeg2dec.c y est\'{a} compuesto por las siguientes funciones principales:

\footnotesize \begin{verbatim}
static void Initialize_Decoder _ANSI_ARGS_((void));
static int Decode_Bitstream _ANSI_ARGS_((void));
\end{verbatim}
\normalsize

La funci\'{o}n de \textit{Decode\_Bitstream} llama a la funci\'{o}n video\_sequence() que se encarga de ir por cada imagen y decodificarlo. \newline

Para ejecutar una demostración, realize los siguientes pasos: \newline

\footnotesize \begin{verbatim}
1) Descomprima el benchmark
   >tar xvzf mediabench.tar.gz

2) Hacer Make 
   >cd mediabench/mpeg2/src/mpeg2dec
   >make

3) Correr sim
   >cd ../../exec/
   >./mpeg2decode.sh

\end{verbatim}
\normalsize

\footnotesize
*Nota: Hay que editar el Makefile y remover la bandera de "-mv8" que viene en la l\'{i}nea de CC. Esta opci\'on ya no es soportado por GCC. \newline
\normalsize

Al ejecutar el script, se deber\'{i}a ver las siguiente l\'{i}neas en la pantalla:

\footnotesize \begin{verbatim}
saving ../data/tmp0.Y
saving ../data/tmp0.U
saving ../data/tmp0.V
saving ../data/tmp1.Y
saving ../data/tmp1.U
saving ../data/tmp1.V
saving ../data/tmp2.Y
saving ../data/tmp2.U
saving ../data/tmp2.V
saving ../data/tmp3.Y
saving ../data/tmp3.U
saving ../data/tmp3.V
\end{verbatim}
\normalsize

Con esto, se logra comprobar que el benchmark funciona correctamente.

\subsection{Herramientas}

\subsubsection{Simplescalar}
Simplescalar\cite{SIMPLESCALAR} es un simulador de arquitectura de computadoras Open Source escrito en C. Este simulador contiene una serie de herramientas que modela un sistema computador virtual con CPU, cache y jerarqu\'{i}as de memoria.
Utilizando Simplescalar, el usuario puede modelar lo que ocurre cuando un programa corre sobre una variedad de arquitecturas desde procesadores sencillos hasta procesadores con scheduling din\'{a}mico, cache no-bloqueante
y con predicci\'{o}n de saltos. Simplescalar soporta set de instrucciones tipo PISA y se compone de tres herramientas principales: \newline

\begin{itemize}
\item \textbf{simplesim} Simulador Simplescalar.
\item \textbf{simpletools} Compilador GNU GCC de PISA para Simplescalar.
\item \textbf{simpleutils} Herramientas para el compilador que incluyen el ensamblador de PISA, linker, etc. \newline
\end{itemize}

\subsubsection{Wattch}
Wattch\cite{WATTCH} es un simulador que estima la potencia consumida de un CPU y otros datos tales como rendimiento de las memorias cache (hits, misses, etc), branch predictors entre otros. Tiene modelos de potencia integrados en Simplescalar y utiliza una version modificada de sim-outorder de Simplescalar para recolectar resultados. \newline

\subsubsection{GCC}
GNU Compiler Collection o GNU C Compiler es un compilador de C para sistemas operativos GNU. \newline

\subsubsection{MPEG2}
Moving Pictures Experts Group 2 (MPEG-2)\cite{MPEG2}, es la designaci\'on para un grupo de est\'andares de codificaci\'on de audio y vídeo acordado por MPEG. MPEG-2 es por lo general usado para codificar audio y video para se\~nales de transmisi\'on, que incluyen televisi\'on digital terrestre, por sat\'elite o cable. Este formato tambi\'en es muy com\'un como formato de codificación usado por los discos SVCD y DVD comerciales de películas.

El algoritmo de compresi\'on general MPEG consiste en aplicar los siguientes pasos\cite{F_MPEG2} \newline

\begin{enumerate}
\item \textbf{Transformaci\'on}. Convierte los datos de entrada en estructuras que permiten ser comprimidos.
\item \textbf{Cuantizaci\'on}. Reduce el n\'umero de s\'imbolos para representar un dato.
\item \textbf{Codificaci\'on de s\'imbolos}. Minimiza la longitud de los s\'imbolos requeridos para representar el dato. \newline
\end{enumerate}

El MPEG-2 permite resoluciones m\'as altas y data rates mayores que MPEG-1.

\subsection{Configuraci\'{o}n}
En esta secci\'{o}n se va a hace un an\'{a}lisis preliminar de la configuraci\'{o}n de la arquitectura (sistema de referencia a utilizar).

El c\'odigo de decodificaci\'on MPEG2 vemos que en su mayor\'ia contiene variables de tipo integer, \
punteros y referencias a memoria para procesar los buffers de datos a decodificar. Existen apenas \
tres variables de tipo double, y se ejecutan unas cuantas operaciones con las mismas.  El codigo se \
basa en llamadas a funciones y luego a otras cuantas mas logrando cerca de unos 5 niveles de \
anidaci\'on de las llamadas, lo cual representan saltos multiples que pueden afectar el IPC. \
Dedido a la muy poca utilizaci\'on de doubles se espera una poca utilizaci\'on de ALUs de punto flotante. \newline 
Ademas de la poca utilizaci\'on de unidades de punto flotante, no se determina algun otro punto crusial en la architectura, por lo cual solamente se modificar\'a las ALUs de FP en la architectura que por default nos brinda WATT.  La misma se presenta a continuaci\'on.

\subsection{Funci\'{o}n Costo}
Definici\'{o}n de la funci\'{o}n costo.
Debido al uso tan extenso que se hace de la codificaci\'{o}n/decodificaci\'{o}n MPEG2, se opta por\
definir nuestra funci\'{o}n de costo en base al rendimiento del chip asi como la energ\'{i}a \
consumida por el mismo, ya que nos parece son los par\'{a}metros que se deber\'{i}an de maximisar para\
esta aplicaci\'{o}n.  El rendimiento es necesario para permitir una decodificaci\'{o}n r\'{a}pida y \
eficiente, la energ\'{i}a es importante por motivos del costo de consumo de la misma para \
realizar las decodificaciones.

\footnotesize \begin{verbatim}
Funcion de costo:     F(x) = P(x) * D(x)
P(x) = Potencia de x (Potencia total consumida 
por el chip)
D(x) = Rendimiento IPC del codigo.
\end{verbatim}
\normalsize

\subsection{Resultados preliminares y Espacio de Dise\~no}
Para obtener los datos preliminares se ejecuta una primer corride del benchmark on la configuraci\'on de Hardware base seleccionada.
De la primer corrida obtenemos los siguientes datos: \newline
\footnotesize \begin{verbatim}
Total Power Consumption: 56.2826
sim_IPC         1.8649 # instructions per cycle
sim_CPI         0.5362 # cycles per instruction
il1.misses      277707 # total number of misses
dl1.misses      96318 # total number of misses
ul2.misses      20691 # total number of misses
itlb.misses     32 # total number of misses
dtlb.misses     114 # total number of misses
il1.hits        178888369 # total number of hits
dl1.hits        32954311 # total number of hits
ul2.hits        384866 # total number of hits
itlb.hit        179166044 # total number of hits
dtlb.hits       33333723 # total number of hits
il1.miss_rate   0.0015 # miss rate (i.e., misses/ref)
dl1.miss_rate   0.0029 # miss rate (i.e., misses/ref)
ul2.miss_rate   0.0510 # miss rate (i.e., misses/ref)
itlb.miss_rate  0.0000 # miss rate (i.e., misses/ref)
dtlb.miss_rate  0.0000 # miss rate (i.e., misses/ref)
\end{verbatim}
\normalsize

Se aprecia que la cantidad de cache misses es muy peque\~na en comparaci\'on con los hits, todos lo miss rates son menores o iguales al 5 por ciento, dada esta caracter\'istica no nos vamos a enfocar en tratart de mejorar el hit rate de las caches, sino en intentar otras configuraciones para intertar bajar la potencia total y el CPI.
Dada la cantidad de saltos se opta por variar las politicas del branch predictor, la configuracion por default ya cuenta con suficientes ALUs para integers por lo que no vamos a intentar variar este componente.  Como se trabaja mucho con datos de memoria y punteros tambien se opta por variar los componenetes para actualizar los registros as\'i como el tama\~no del queue para los Loads y Stores.

Nuestro espacio de dise\~no va a ser entonces: branch preditor, el register update unit (RUU) size, load/store queue (LSQ) size, y la politica de reemplazo de bloques en las memorias cache(FIFO, LFU).

\subsection{Simulaciones a realizar}
C\'{a}lculo de la cantidad total de simulaciones a realizar.
Las simulaciones preliminares nos dan un estimado de 279 segundos para correr cada simulaci\'{o}n, \
lo cual corresponde aproximadamente a 4 minutos.  En un lapso de 45 minutos teoricamente se pueden correr alrededor de 11 simulaciones. Dado que son pocas las iteraciones que se pueden correr en un lapso de 45 minutos,
nos tenemos que limitar a correr 12 simulaciones, lo que implica que solo se van a probar 12 combinaciones posibles.
A continuaci\'{o}n se muestra las pruebas y su configuracion: \newline

Configuraci\'on base:
\footnotesize \begin{verbatim}
bpred           comb
ruu:size        16
lsq:size        8
cache:dl1       dl1:128:32:4:l
cache:dl2       ul2:1024:64:4:l
cache:il1       il1:512:32:1:l
cache:il2       dl2
res:ialu        4
res:imult       1
res:fpalu       1
res:fpmult      1
\end{verbatim}
\normalsize

Detalles de los tests: \newline

\begin{enumerate}
\item \textbf{Test 1}.  bpredm=comb, lsq=8, ruu=16, cache=lfu
\item \textbf{Test 2}.  bpredm=comb, lsq=8, ruu=32, cache=lfu
\item \textbf{Test 3}.  bpredm=comb, lsq=8, ruu=32, cache=fifo
\item \textbf{Test 4}.  bpredm=comb, lsq=16, ruu=16, cache=lfu
\item \textbf{Test 5}.  bpredm=comb, lsq=16, ruu=32, cache=lfu
\item \textbf{Test 6}.  bpredm=comb, lsq=16, ruu=32, cache=fifo
\item \textbf{Test 7}.  bpredm=2lev, lsq=8, ruu=16, cache=lfu
\item \textbf{Test 8}.  bpredm=2lev, lsq=8, ruu=32, cache=lfu
\item \textbf{Test 9}.  bpredm=2lev, lsq=8, ruu=32, cache=fifo
\item \textbf{Test 10}. bpredm=2lev, lsq=16, ruu=16, cache=lfu
\item \textbf{Test 11}. bpredm=2lev, lsq=16, ruu=32, cache=lfu
\item \textbf{Test 12}. bpredm=2lev, lsq=16, ruu=32, cache=fifo \newline
\end{enumerate}

Antes de lanzar las simulaciones, es impotante notar que se requiere volver a make de los binarios (mpeg2decode) haciendo una compilaci\'on cruzada con \textit{simplescalar/bin/sslittle-na-sstrix-gcc}. Esto se logra editando
el Makefile dentro de \textit{src/mpeg2dec/} y cambiar el CC\footnote{Es necesario remover el flag -lm}.
Para lanzar las simulaciones, se ejecuta el script \textit{./run\_benchmark.sh} en el directorio \textit{src/mpeg2dec/}. Este script se encarga de llamar a Wattch junto con el binario cross-compilado de Simplescalar.

\subsection{Resultados y An\'alisis}

\begin{figure}[!ht]
        \begin{center}
        \includegraphics[width=3in]{fig1.png}
        \caption{Resultados de Potencia total}
        \end{center}
\end{figure}

\begin{figure}[!ht]
        \begin{center}
        \includegraphics[width=3in]{fig2.png}
        \caption{Resultados de IPC}
        \end{center}
\end{figure}

\begin{figure}[!ht]
        \begin{center}
        \includegraphics[width=3in]{fig3.png}
        \caption{Gr\'{a}fica de Funci\'{o}n Costo}
        \end{center}
\end{figure}

A partir de los resultados se observa que al reducir el tama\~{n}o de la RUU se obtiene mejor potencia total (m\'{a}s baja). Esto es de imaginar ya que el algoritmo de decodificaci\'{o}n de mpeg2 hace bastante uso de de buffers y de operaciones en registros. Al mismo tiempo, al reducir la RUU de 32 a 16, aumenta la cantidad de ciclos necesarios para ejecutar una instrucci\'{o}n (CPI). Noten que este aumento en la CPI
tiene mayor peso sobre la funci\'{o}n costo que la potencia reducida. En otras palabras, la mejora que se obtuvo en potencia no fue lo suficiente para compensar el incremento en la CPI. Al incrementar la CPI se incrementa tambi\'{e}n
el tiempo de ejecuci\'{o}n del programa lo cual no es deseable.

El segundo factor que mas influenci\'{o} fue el modo del Branch Predictor, y las variables utilizadas son \textit{2Lev} y \textit{comb} con el \'{u}ltimo siendo el m\'{e}todo por default del compilador. Note con el m\'{e}todo \textit{comb} se produce
mejores tasas de CPI de forma consistente en comparaci\'{o}n con \textit{2Lev}.

Los cambios en la estrategia de reemplazo de los bloques de la cache y el tama\~no del queue de LoadStore no aportan variaciones significativas de la funci\'{o}n costo aunque se observa una ligera mejora en la CPI al utilizar LFU en comparaci\'{o}n con utilizar FIFO. Esto se debe a que el c\'{o}digo hace un uso intensivo de la memoria y las acciones realizadas con estos datos permiten calendarizar otras operaciones entre lecturas de memoria.


\section{Conclusiones}
El tama\~no f\'isico de las RUUs tienen mucho impacto sobre el desempe\~no. Si el RUU es relativamente peque\~no el CPI se va a verse impactado negativamente produciendo un pobre resultado de funci\'on costo, eclipsando
mejora en consumo de potencia que pudo haber traer el reducir el tama\~no de las RUUs.

El branch predictor tiene un peque\~no impacto en el CPI y en la funci\'on costo. Se encontr\'o que el tipo de predicci\'on \textit{comb} tiene mejor desempe\~no que \textit{2lev}. \newline

Configuraci\'on \'optima: \newline

\footnotesize \begin{verbatim}
bpred           comb
ruu:size        32
lsq:size        8
cache:dl1       dl1:128:32:4:l
cache:dl2       ul2:1024:64:4:l
cache:il1       il1:512:32:1:l
cache:il2       dl2
res:ialu        4
res:imult       1
res:fpalu       1
res:fpmult      1
\end{verbatim}
\normalsize

% use section* for acknowledgement
\section*{Acknowledgment}

\begin{thebibliography}{1}
\bibitem{MPEG2-ENCODER/DECODER}
MPEG Software Simulation Group, MPEG-2 Encoder/Decoder, Version 1.2, July 19, 1996

\bibitem{SIMPLESCALAR}
Simplescalar LLC, http://www.simplescalar.com/, 2395 Timbercrest Court Ann Arbor, MI 48105

\bibitem{WATTCH}
Wattch, http://www.eecs.harvard.edu/~dbrooks/wattch-form.html, Version 1.02d.

\bibitem{MPEG2}
"MPEG-2", "http://en.wikipedia.org/wiki/MPEG-2", Wikipedia.

\bibitem{F_MPEG2}
Haitzler, Carsten; Webber, Nicholas, "MPEG(2) Encoding \& Decoding, "http://www.fst.net/resources/papers/Introduction+to+MPEG
+Encoding+and+Decoding.pdf",Fluffy Spider Technologies Pty Ltd,Suite 87,330 Wattle Street,Ultimo NSW, 2007

\end{thebibliography}

\end{document}


