\documentclass[journal]{IEEEtran}
% *** GRAPHICS RELATED PACKAGES ***
\ifCLASSINFOpdf
  \usepackage[pdftex]{graphicx}
\else
  \usepackage[dvips]{graphicx}
\fi

% *** MATH PACKAGES ***
\usepackage[cmex10]{amsmath}

\begin{document}
%
% paper title
\title{Decodificaci\'{o}n MPEG2}
\author{Alberto~Suarez,~\IEEEmembership{Ingeniero,~Hewlett~Packard,}
        Yu~Shan~Hsieh,~\IEEEmembership{Ingeniero,~Hewlett~Packard}% <-this % stops a space
\thanks{Cristina Murillo, ITCR}% <-this % stops a space
}
% make the title area
\maketitle

% in the abstract or keywords.
\begin{abstract}
Art\'{i}culo cient\'{i}fico sobre benchmarking en la decodificaci\'{o}n de mpeg2.
\end{abstract}

% Note that keywords are not normally used for peerreview papers.
\begin{IEEEkeywords}
Decodificaci\'{o}n, mpeg2, Benchmark
\end{IEEEkeywords}

\section{Introducci\'{o}n}
\IEEEPARstart{E}{ste} documento va a hablar sobre el benchmarking de la decodificaci\'{o}n de mpeg2.

\hfill Setiembre 27, 2013

\section{Metodolog\'{i}a de optimizaci\'{o}n}

\subsection{Descripci\'{o}n del benchmark}
Descripci\'{o}n del benchmark

\subsection{Herramientas}
Herramientas utilizadas para la simulaci\'{o}n.

\subsection{Configuraci\'{o}n}
An\'{a}lisis preliminar de la configuraci\'{o}n de la arquitectura (sistema de referencia a utilizar).

El c\'odigo de decodificaci\'on MPEG2 vemos que contiene en su mayor\'ia contiene variables de tipo integer, punteros y referencias a memoria para procesar los buffers de datos a decodificar. Existen apenas tres variables de tipo double, y se ejecutan unas cuantas operaciones con las mismas.  El codigo se basa en llamadas a funciones y luego a otras cuantas mas logrando cerca de unos 5 niveles de anidaci\'on de las llamadas, lo cual representan saltos multiples que pueden afectar el IPC.
Dedido a la muy poca utilizaci\'on de doubles se espera una poca utilizaci\'on de ALUs de punto flotante. \newline

\subsection{Simulaciones a realizar}
C\'{a}lculo de la cantidad total de simulaciones a realizar.
Las simulaciones preliminares nos dan un estimado de 279 segundos para correr cada simulaci\'{o}n, \
lo cual corresponde aproximadamente a 4 minutos.  En un lapso de 45 minutos teoricamente se pueden correr alrededor de 11 simulaciones.


\subsection{Funci\'{o}n Costo}
Definici\'{o}n de la funci\'{o}n costo.
Debido al uso tan extenso que se hace de la codificaci\'{o}n/decodificaci\'{o}n MPEG2, se opta por\
definir nuestra funci\'{o}n de costo en base al rendimiento del chip asi como la energ\'{i}a \
consumida por el mismo, ya que nos parece son los par\'{a}metros que se deber\'{i}an de maximisar para\
esta aplicaci\'{o}n.  El rendimiento es necesario para permitir una decodificaci\'{o}n r\'{a}pida y \
eficiente, la energ\'{i}a es importante por motivos del costo de consumo de la misma para \
realizar las decodificaciones.

Funci\'{o} de costo:     F(x) = P(x) * D(x)
P(x) = Potencia de x (Potencia total consumida por el chip)
D(x) = Rendimiento IPC del codigo.

\subsection{EspacioDiseno}
Resultados preliminares y Espacio de Dise\~no.
De la primer corrida obtenemos los siguientes datos:
Total Power Consumption: 56.2826
sim\_IPC                      1.8649 # instructions per cycle
sim\_CPI                      0.5362 # cycles per instruction
il1.misses                   277707 # total number of misses
dl1.misses                    96318 # total number of misses
ul2.misses                    20691 # total number of misses
itlb.misses                      32 # total number of misses
dtlb.misses                     114 # total number of misses
il1.hits                  178888369 # total number of hits
dl1.hits                   32954311 # total number of hits
ul2.hits                     384866 # total number of hits
itlb.hits                 179166044 # total number of hits
dtlb.hits                  33333723 # total number of hits
il1.miss\_rate                0.0015 # miss rate (i.e., misses/ref)
dl1.miss\_rate                0.0029 # miss rate (i.e., misses/ref)
ul2.miss\_rate                0.0510 # miss rate (i.e., misses/ref)
itlb.miss\_rate               0.0000 # miss rate (i.e., misses/ref)
dtlb.miss\_rate               0.0000 # miss rate (i.e., misses/ref)

Se aprecia que la cantidad de cache misses es muy peque\~na en comparaci\'on con los hits, todos lo miss rates son menores o iguales al 5 por ciento, dada esta caracter\'istica no nos vamos a enfocar en tratart de mejorar el hit rate de las caches, sino en intentar otras configuraciones para intertar bajar la potencia total y el CPI.
Dado que son pocas las iteraciones que se pueden correr en un lapso de 45 minutos, nos tenemos que limitar a correr 12 simulaciones, lo que implica que solo se van a probar 12 combinaciones posibles.  Dada la cantidad de saltos se opta por variar las politicas del branch predictor, la configuracion por default ya cuenta con suficientes ALUs para integers por lo que no vamos a intentar variar este componente.  Como se trabaja mucho con datos de memoria y punteros tambien se opta por variar los componenetes para actualizar los registros as\'i como el tama\~no del queue para los Loads y Stores.

Nuestro espacio de dise\~no va a ser entonces: branch preditor, el register update unit (RUU) size, load/store queue (LSQ) size, y la politica de reemplazo de bloques en las memorias cache.

\subsection{Resultados}
Resultados obtenidos.

\subsection{An\'{a}lisis de Resultados}
An\'{a}lisis de resultados: analizar el punto \'{o}ptimo con base al c\'{o}digo del benchmark, comparaci\'{o}n con el sistema de referencia.

\section{Conclusiones}
Conclusiones.

% use section* for acknowledgement
\section*{Acknowledgment}

\begin{thebibliography}{1}

\bibitem{MPEG2-ENCODER/DECODER}
MPEG Software Simulation Group, MPEG-2 Encoder/Decoder, Version 1.2, July 19, 1996
\end{thebibliography}

\begin{IEEEbiographynophoto}{Alberto Suarez}
Ingeniero Electr\'{o}nico, Hewlett Packard
\end{IEEEbiographynophoto}

% if you will not have a photo at all:
\begin{IEEEbiographynophoto}{Yu Shan Hsieh}
Bachillerato en Ingenier\'{i}a El\'{e}ctrica, UCR
Ingeniero Electr\'{o}nico, Hewlett Packard
\end{IEEEbiographynophoto}

\end{document}


