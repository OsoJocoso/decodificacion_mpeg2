\documentclass[journal]{IEEEtran}
% *** GRAPHICS RELATED PACKAGES ***
\ifCLASSINFOpdf
  \usepackage[pdftex]{graphicx}
\else
  \usepackage[dvips]{graphicx}
\fi

% *** MATH PACKAGES ***
\usepackage[cmex10]{amsmath}

\begin{document}
%
% paper title
\title{Decodificaci\'{o}n MPEG2}
\author{Alberto~Suarez,~\IEEEmembership{Ingeniero,~Hewlett~Packard,}
        Yu~Shan~Hsieh,~\IEEEmembership{Ingeniero,~Hewlett~Packard}% <-this % stops a space
\thanks{}% <-this % stops a space
}
% make the title area
\maketitle

% in the abstract or keywords.
\begin{abstract}
Art\'{i}culo cient\'{i}fico sobre benchmarking en la decodificaci\'{o}n de mpeg2.
\end{abstract}

% Note that keywords are not normally used for peerreview papers.
\begin{IEEEkeywords}
Decodificaci\'{o}n, mpeg2, Benchmark
\end{IEEEkeywords}

\section{Introducci\'{o}n}
\IEEEPARstart{E}{l} prop\'{o}sito de este paper es analizar varios aspectos de rendimiento que tiene un programa de decodificaci\'{o}n de mpeg2 tales como consumo de potencia, desempeño de la CPU, tiempo de ejecuci\'{o}n, etc. 
Para poder hablar sobre esos aspectos, es importante primero entender qu\'{e} es la decodificaci\'{o}n de mpeg2 y c\'{o}mo se realiza. Para esto se analizar\'{a} los c\'{o}digos fuentes que  Posteriormente se detallar\'{a} las pruebas que se van a realizar y el testbench sobre la cual se realizar\'{a} las pruebas. Luego que har\'{a} el an\'{a}lisis de resultados y finalmente las conclusiones que se pueden derivar de estos experimentos.


\hfill Setiembre 27, 2013

\section{Metodolog\'{i}a de optimizaci\'{o}n}

\subsection{Descripci\'{o}n del benchmark}

\subsection{Herramientas}

\subsubsection{Simplescalar}
Simplescalar\cite{SIMPLESCALAR} es un simulador de arquitectura de computadoras Open Source escrito en C. Este simulador contiene una serie de herramientas que modela un sistema computador virtual con CPU, cache y jerarqu\'{i}as de memoria.
Utilizando Simplescalar, el usuario puede modelar lo que ocurre cuando un programa corre sobre una variedad de arquitecturas desde procesadores sencillos hasta procesadores con scheduling din\'{a}mico, cache no-bloqueante
y con predicci\'{o}n de saltos. Simplescalar soporta set de instrucciones tipo PISA y se compone de tres herramientas principales: \newline

\begin{itemize}
\item \textbf{simplesim} Simulador Simplescalar.
\item \textbf{simpletools} Compilador GNU GCC de PISA para Simplescalar.
\item \textbf{simpleutils} Herramientas para el compilador que incluyen el ensamblador de PISA, linker, etc. \newline
\end{itemize}

\subsubsection{Wattch}
Wattch\cite{WATTCH} es un simulador que estima la potencia consumida de un CPU. Tiene modelos de potencia integrados en Simplescalar y utiliza una version modificada de sim-outorder de Simplescalar para recolectar resultados. \newline

\subsubsection{GCC}
GNU Compiler Collection o GNU C Compiler es un compilador de C para sistemas operativos GNU. \newline

\subsubsection{Mpeg2 decoder}
El decodificador viene en un suite que trabaja tanto la parte de codificaci\'{o}n (encode) como la de decodificaci\'{o}n. En nuestro caso, solo vamos a trabajar la parte de decode.
Dentro del directorio de src/mpeg2dec se encuentra el codigo en C del decoder junto con el Makefile. El archivo principal es mpeg2dec.c y est\'{a} compuesto por las siguientes funciones principales:

\footnotesize \begin{verbatim}
static void Initialize_Decoder _ANSI_ARGS_((void));
static int Decode_Bitstream _ANSI_ARGS_((void));
\end{verbatim}
\normalsize

La funci\'{o}n de \textit{Decode\_Bitstream} llama a la funci\'{o}n video\_sequence() que se encarga de ir por cada imagen y decodificarlo. \newline

\subsection{Configuraci\'{o}n}
An\'{a}lisis preliminar de la configuraci\'{o}n de la arquitectura (sistema de referencia a utilizar).

El c\'odigo de decodificaci\'on MPEG2 vemos que contiene en su mayor\'ia contiene variables de tipo integer, punteros y referencias a memoria para procesar los buffers de datos a decodificar. Existen apenas tres variables de tipo double, y se ejecutan unas cuantas operaciones con las mismas.  El codigo se basa en llamadas a funciones y luego a otras cuantas mas logrando cerca de unos 5 niveles de anidaci\'on de las llamadas, lo cual representan saltos multiples que pueden afectar el IPC.
Dedido a la muy poca utilizaci\'on de doubles se espera una poca utilizaci\'on de ALUs de punto flotante. \newline

\subsection{Simulaciones a realizar}
C\'{a}lculo de la cantidad total de simulaciones a realizar.
Las simulaciones preliminares nos dan un estimado de 279 segundos para correr cada simulaci\'{o}n, \
lo cual corresponde aproximadamente a 4 minutos.  En un lapso de 45 minutos teoricamente se pueden correr alrededor de 11 simulaciones.


\subsection{Funci\'{o}n Costo}
Definici\'{o}n de la funci\'{o}n costo.
Debido al uso tan extenso que se hace de la codificaci\'{o}n/decodificaci\'{o}n MPEG2, se opta por\
definir nuestra funci\'{o}n de costo en base al rendimiento del chip asi como la energ\'{i}a \
consumida por el mismo, ya que nos parece son los par\'{a}metros que se deber\'{i}an de maximisar para\
esta aplicaci\'{o}n.  El rendimiento es necesario para permitir una decodificaci\'{o}n r\'{a}pida y \
eficiente, la energ\'{i}a es importante por motivos del costo de consumo de la misma para \
realizar las decodificaciones.

\footnotesize \begin{verbatim}
Funcion de costo:     F(x) = P(x) * D(x)
P(x) = Potencia de x (Potencia total consumida por el chip)
D(x) = Rendimiento IPC del codigo.
\end{verbatim}
\normalsize

\subsection{EspacioDiseno}
Resultados preliminares y Espacio de Dise\~no.


\subsection{Resultados}
Resultados obtenidos.

\subsection{An\'{a}lisis de Resultados}
An\'{a}lisis de resultados: analizar el punto \'{o}ptimo con base al c\'{o}digo del benchmark, comparaci\'{o}n con el sistema de referencia.

\section{Conclusiones}
Conclusiones.

% use section* for acknowledgement
\section*{Acknowledgment}


El autor desea agradecer a la profesora Cristina Murillo por ser tan linda ;)

\begin{thebibliography}{1}
\bibitem{MPEG2-ENCODER/DECODER}
MPEG Software Simulation Group, MPEG-2 Encoder/Decoder, Version 1.2, July 19, 1996

\bibitem{SIMPLESCALAR}
Simplescalar LLC, http://www.simplescalar.com/, 2395 Timbercrest Court Ann Arbor, MI 48105

\bibitem{WATTCH}
Wattch, http://www.eecs.harvard.edu/~dbrooks/wattch-form.html, Version 1.02d.

\end{thebibliography}

\begin{IEEEbiographynophoto}{Alberto Suarez}
Ingeniero Electr\'{o}nico, Hewlett Packard
\end{IEEEbiographynophoto}

% if you will not have a photo at all:
\begin{IEEEbiographynophoto}{Yu Shan Hsieh}
Bachillerato en Ingenier\'{i}a El\'{e}ctrica, UCR
Ingeniero Electr\'{o}nico, Hewlett Packard
\end{IEEEbiographynophoto}

\end{document}


