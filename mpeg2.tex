\documentclass[journal]{IEEEtran}
% *** GRAPHICS RELATED PACKAGES ***
\ifCLASSINFOpdf
  \usepackage[pdftex]{graphicx}
\else
  \usepackage[dvips]{graphicx}
\fi

% *** MATH PACKAGES ***
\usepackage[cmex10]{amsmath}

\begin{document}
%
% paper title
\title{Decodificaci\'{o}n MPEG2}
\author{Alberto~Suarez,~\IEEEmembership{Ingeniero,~Hewlett~Packard,}
        Yu~Shan~Hsieh,~\IEEEmembership{Ingeniero,~Hewlett~Packard}% <-this % stops a space
\thanks{Cristina Murillo, ITCR}% <-this % stops a space
}
% make the title area
\maketitle

% in the abstract or keywords.
\begin{abstract}
Art\'{i}culo cient\'{i}fico sobre benchmarking en la decodificaci\'{o}n de mpeg2.
\end{abstract}

% Note that keywords are not normally used for peerreview papers.
\begin{IEEEkeywords}
Decodificaci\'{o}n, mpeg2, Benchmark
\end{IEEEkeywords}

\section{Introducci\'{o}n}
\IEEEPARstart{E}{ste} documento va a hablar sobre el benchmarking de la decodificaci\'{o}n de mpeg2.

\hfill Setiembre 27, 2013

\section{Metodolog\'{i}a de optimizaci\'{o}n}

\subsection{Descripci\'{o}n del benchmark}
Descripci\'{o}n del benchmark

\subsection{Herramientas}
Herramientas utilizadas para la simulaci\'{o}n.

\subsection{Configuraci\'{o}n}
An\'{a}lisis preliminar de la configuraci\'{o}n de la arquitectura (sistema de referencia a utilizar).

El c\'odigo de decodificaci\'on MPEG2 vemos que contiene en su mayor\'ia contiene variables de tipo integer, punteros y referencias a memoria para procesar los buffers de datos a decodificar. Existen apenas tres variables de tipo double, y se ejecutan unas cuantas operaciones con las mismas.  El codigo se basa en llamadas a funciones y luego a otras cuantas mas logrando cerca de unos 5 niveles de anidaci\'on de las llamadas, lo cual representan saltos multiples que pueden afectar el IPC.
Dedido a la muy poca utilizaci\'on de doubles se espera una poca utilizaci\'on de ALUs de punto flotante. \newline

\subsection{Simulaciones a realizar}
C\'{a}lculo de la cantidad total de simulaciones a realizar.
Las simulaciones preliminares nos dan un estimado de 279 segundos para correr cada simulaci\'{o}n, \
lo cual corresponde aproximadamente a 4 minutos.  En un lapso de 45 minutos teoricamente se pueden correr alrededor de 11 simulaciones.


\subsection{Funci\'{o}n Costo}
Definici\'{o}n de la funci\'{o}n costo.
Debido al uso tan extenso que se hace de la codificaci\'{o}n/decodificaci\'{o}n MPEG2, se opta por\
definir nuestra funci\'{o}n de costo en base al rendimiento del chip asi como la energ\'{i}a \
consumida por el mismo, ya que nos parece son los par\'{a}metros que se deber\'{i}an de maximisar para\
esta aplicaci\'{o}n.  El rendimiento es necesario para permitir una decodificaci\'{o}n r\'{a}pida y \
eficiente, la energ\'{i}a es importante por motivos del costo de consumo de la misma para \
realizar las decodificaciones.

Funci\'{o} de costo:     F(x) = P(x) * D(x)
P(x) = Potencia de x (Potencia total consumida por el chip)
D(x) = Rendimiento IPC del codigo.

\subsection{EspacioDiseno}
Resultados preliminares y Espacio de Dise\~no.


\subsection{Resultados}
Resultados obtenidos.

\subsection{An\'{a}lisis de Resultados}
An\'{a}lisis de resultados: analizar el punto \'{o}ptimo con base al c\'{o}digo del benchmark, comparaci\'{o}n con el sistema de referencia.

\section{Conclusiones}
Conclusiones.

% use section* for acknowledgement
\section*{Acknowledgment}


El autor desea agradecer a la profesora Cristina Murillo por ser tan linda ;)

\begin{thebibliography}{1}

\bibitem{MPEG2-ENCODER/DECODER}
MPEG Software Simulation Group, MPEG-2 Encoder/Decoder, Version 1.2, July 19, 1996
\end{thebibliography}

\begin{IEEEbiographynophoto}{Alberto Suarez}
Ingeniero Electr\'{o}nico, Hewlett Packard
\end{IEEEbiographynophoto}

% if you will not have a photo at all:
\begin{IEEEbiographynophoto}{Yu Shan Hsieh}
Bachillerato en Ingenier\'{i}a El\'{e}ctrica, UCR
Ingeniero Electr\'{o}nico, Hewlett Packard
\end{IEEEbiographynophoto}

\end{document}


